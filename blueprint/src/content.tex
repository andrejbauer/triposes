\section{Definition of a tripos}

\begin{definition}[Type-Tripos]
    A \emph{Type-tripos} is a functor \(P : \ty^\op \to \ha\) between poset-enriched categories \(\ty\) of types and \(\ha\) of heyting algebras, that further satisfies the following:
    \begin{enumerate}
        \item For each types \(X\) and \(Y\), and \(f : X \to Y\) the map \(Pf\) has monotone left and right adjoints, \(\exists_f\) and \(\forall_f\) respectively
        \item These further satisfy the Beck-Chevalley condition
        %\item For each type \(X\) there is a type \(\pi(X)\) and an element \(\in_X \in P(X\prod\pi(X))\) satisfying ...
        \item There is a type \(\Sigma\) and an element \(\sigma \in P(\Sigma)\) such that for every type \(X\) and every \(\phi \in P(X)\) there is a morphism \([\phi] : X \to \Sigma\) such that \(\phi = P([\phi])(\sigma)\)
    \end{enumerate}
\end{definition}

How to implement this is not at all clear. I think most of the work will actually go into defining this.

\section{Definition of a partial equivalence relation}

\begin{definition}
    A \emph{partial equivalence relation} on a type \(X\) over a Type-tripos \(P\) is a map \([-=-]: X×X → P(X)\) that satisfies
    \begin{enumerate}
        \item \([a=b] ≤ [b=a]\)
        \item \([a=b]∧[b=c] ≤ [a=c]\)
    \end{enumerate}
\end{definition}

In ``the internal language'' the first condition is symmetry and the second is transitivity. So a partial equivalence relation over \(P\) is a ``symmetric transitive relation internal to some language in \(P\)''.

\begin{definition}
    A morphism between types \(X\) and \(Y\) along with PERs \([-=_X-]\) and \([-=_Y-]\) over \(P\) is a map \([f(-)=-] : X×Y → P(Y)\) such that
    \begin{enumerate}
        \item \([a=_Xa']∧[f(a')=b]≤ [f(a) = b]\)
        \item \([f(a) = b]∧[b=_Yb'] ≤ [f(a) = b']\)
        \item \([f(a)=b]∧[f(a)=b'] ≤ [b=_Yb']\)
        \item \([a=_Xa] ≤ ∃_{π} [f(a) = b]\)
    \end{enumerate}
\end{definition}

\begin{remark}
    The \(π\) in the fourth condition is the projection map \(X×Y → Y\).
\end{remark}

Again, the first two properties here are coherences with the PERs on the domain and codomain, and the other two are uniqueness and totality of a relation. So functions are defined as ``functional relations internal to \(P\)''.

\section{Tripos to topos construction}

\begin{definition}
    To a Type-tripos \(P\) we can associate the category \(\ty[P]\) of types along with partial equivalence relations over \(P\) with morphisms as above.
\end{definition}

\begin{theorem}[Pitts]
    The category \(\ty[P]\) is a topos.
\end{theorem}


\section{Definition of a tripos language}

TBD once I settle whether I can do the above in a general category instead of \(\ty\).

\section{Tripos to topos construction bis.}


